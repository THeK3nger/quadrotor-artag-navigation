\section{Navigation}

\subsection{The navigation problem}

Now that our quadrotor knows how to do hovering on a single point, it is time to
add more tags to make possible for it to follow a path.

The quadrotor maintains a list of the existent tags ($S = (s_1,...,s_n)$) and
a pointer (usually an integer index) to the current active tag. At any time we
can order a change for quadrotor active tag and see it moves itself towards the 
selected new active one.

In our simulation we use a simple sequential path: the robot can follow only
one direction: forward or backward. In general it is possible to create more
complex path (for example a tree-like path) with the only limitation that the 
quadrotor camera must see every possible next tag.

\subsection{Tag switch sequence}

The most critical aspect of tag navigation is the \emph{tag switching}
sequence. This sequence is composed by three steps:

\begin{enumerate}
    \item Search for the next tag. Suppose that the quadrotor active tag is
        $s_i$ and the next desired tag is $s_j$. In that case the algorithm
        will search for tag $s_j$ in the camera image still using $s_i$ as
        reference tag.
    \item Switch the active target to $s_j$ updating the desired position
        according to the offset between $s_i$ and $s_j$. This is important
        because switching the active target changes the reference frame of the
        quadrotor and introduces a sensible instant error in the controller
        input. The new desired position is chosen such that position error is
        zero in the switch instant.
    \item Drag the desired position back to its original value.
\end{enumerate}

In our project the point 3 is achieved with a simple linear interpolation:

\begin{equation}
    \boldsymbol{d}(t) = \boldsymbol{d}_f \frac{t-t_0}{T} + (1-\frac{t-t_0}{T})\boldsymbol{d}_i\:\:\: \text{for}\, t_0 \le t \le t_0+T
\end{equation}

where $t_0$ is the instant in which the switching begin, $T$ is the switching
duration, and $\boldsymbol{d}_i$ and $\boldsymbol{d}_f$ is the starting and
final position.
