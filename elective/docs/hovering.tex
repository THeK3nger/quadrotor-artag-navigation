\section{Hovering}

\subsection{The hovering problem}

As the first requirement for this project, the quadrotor has to be able to
perform hovering on a single tag. To do this the quadrotor can use only these
data:

\begin{itemize} 
    \item The vector $\hat{x},\hat{y},\hat{z}$ of the estimated position of the
        quadrotor respect to the current active tag\footnote{In the hovering
        problem the active tag corresponds to the only available tag.}.
    \item The vector $\hat{v}_x, \hat{v}_y, \hat{v}_z$ of the estimated
        velocities of the quadrotor. These values are computed by a combination
        of IMU sensor data and an optical flow on the camera input.
\end{itemize}

The quadrotor must use this input to stay in a fixed position whit respect to the
current active tag. In fact we assume that the desired position is expressed in
the active tag reference frame regardless of the real position in world
coordinates 

This is an important assumption because it greatly simplify the equation
involved and, in addition, it allows to perform target hovering also when we do
not know the real position of the tag in the world frame.

\subsection{Position estimation}

As we said the main sensor input for the quadrotor localization comes from the
ARTag pose estimation of the camera. These values represent an approximation
of the real quadrotor position in a reference frame centered in the active tag.

However the input of ARTag cannot be used directly as reference for the PD
controller of the quadrotor. The ARTag estimation are very noisy an subject to
sensible variations at each frame. To have a more robust estimation we perform a 
low-pass filtering (using an exponential filter) on the ARTag current input
$\hat{\boldsymbol{p}}^{\star}_{t}$.

\begin{equation}
    \hat{\boldsymbol{p}}_{t} = \hat{\boldsymbol{p}}_{t-1} (1-e^{k\Delta{\boldsymbol{p}}}) + \hat{\boldsymbol{p}}^{\star}_{t}(e^{k\Delta{\boldsymbol{p}}})
\end{equation}

where

\begin{equation}
    \Delta{\boldsymbol{p}} = (\hat{\boldsymbol{p}}_{t-1} - \hat{\boldsymbol{p}}_t^\star)^2
\end{equation}

and $k$ is just a real negative constant called \emph{smooth factor}.
